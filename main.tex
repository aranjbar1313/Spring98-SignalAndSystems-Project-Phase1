\documentclass{utsignal}

\usepackage{amsmath}
\usepackage{amssymb}
\usepackage{wrapfig}
\usepackage{verbatim}
\usepackage{fancyvrb}
\usepackage{lscape}
\usepackage{rotating}
\usepackage{xepersian}
\usepackage{listings}
\usepackage{color}
\usepackage[utf8]{inputenc}

\title{پروژه - فاز ۱}
\course{سیگنال‌ها و سیستم‌ها}
\author{\href{mailto:h.barkhordarpour@ut.ac.ir?subject=[SS\%20S98 A2]}{هدی برخوردارپور}، 
\href{mailto:ranjbar.ali@ut.ac.ir?subject=[SS\%20S98 A2]\%20}{علی رنجبر}}
%\lecturer{امیرمسعود ربیعی}
\deadline{جمعه ۲۳ فروردین ۱۳۹۸، ساعت ۲۳:۵۵}
\graphicspath{{./img/}}


\begin{document}
	\maketitle
	\section*{کلیات پروژه}
	در این پروژه می‌خواهیم برنامه‌ای مانند شازم \LTRfootnote{Shazam}برای جستجوی قطعه‌هایی از آهنگ در بین پایگاه داده‌ای از اهنگ‌ها بنویسیم. روش اصلی به صورت زیر است:
	\begin{enumerate}
		\item ساخت یک پایگاه داده از ویژگی‌های هر آهنگ با طول کامل
		\item استخراج ویژگی‌های متناظر برای قطعه‌ی آهنگ  (کلیپ صوتی)که باید در بین آهنگ‌های پایگاه داده شناسایی شود.
		\item جستجوی پایگاه داده برای یافتن مطابقت بین ویژگی‌‌های قطعه آهنگ با ویژگی‌های ذخیره شده در پایگاه داده
	\end{enumerate}
	مانند شازم، ویژگی‌های هر آهنگ (یا کلیپ صوتی) جفت پیک‌های مجاور در اسپکتروگرام آهنگ (یا کلیپ صوتی) خواهد بود. پس با پیدا کردن پیک‌های اسپکتروگرام شروع می‌کنیم و با رسم محل این پیک‌ها به دیاگرام تصمیم گیری \LTRfootnote{Constellation map} می‌رسیم. هر پیک در یک زمان و فرکانس مشخص	 \lr{(t, f)} اتفاق می‌افتد و یک دامنه \lr{(A)} دارد. سپس جفت پیک‌هایی که در فاصله‌ی زمانی و فرکانسی مشخصی از هم قرار دارند، در پایگاه داده ذخیره می‌کنیم. برای مثال، اگر دو پیک در $(f_1, t_1)$ و $(f_2, t_2)$ داشته باشیم، ممکن است $(f_1, f_2, t_1, t_2-t_1,A_1,A_2,songid)$ را ذخیره کنیم. اما نمی‌خواهیم همه‌ی جفت پیک‌های ممکن را ذخیره کنیم چون تعداد آن‌ها بسیار زیاد است. بعدا نحوه‌ی انتخاب این پیک‌ها را توضیح می‌دهیم.
	
	بجای $(f_1, _f2, t_1, t_2-t_1,A_1,A_2,songid)$، اندازه‌ها را حذف کرده و فقط $(f_1, f_2, t_1, t_2-t_1,songid)$ را ذخیره می‌کنیم. دلیل این کار در مقاله بیان شده است. به صورت خلاصه: ممکن است دامنه‌ی پیک نسبت به تغییرات بهره در فرکانس، مقاوم نباشد.
	
	هر آهنگ به صورت یک جدول از ویژگی‌ها خلاصه می‌شود. این خلاصه مانند یک اثر انگشت برای آهنگ است.
	\begin{center}
		\begin{latin}
			\begin{tabular}{|c|c|c|c|c|}
				$f_1$ & $f_2$ & $t_1$ & $t_2-t_1$ & songid\\
				&&&\vdots&\\
				$f_j$ & $f_k$ & $t_j$ & $t_k-t_j$ & songid\\
				&&&\vdots&\\
				$f_m$ & $f_n$ & $t_m$ & $t_n-t_m$ & songid\\
			\end{tabular}
		\end{latin}
	\end{center}
	برای هر کلیپ صوتی نیز همان مراحل استخراج ویژگی‌ها را انجام می‌دهیم و جدولی مانند زیر بدست می‌آوریم:
	\begin{center}
		\begin{latin}
			\begin{tabular}{|c|c|c|c|}
				$f_1$ & $f_2$ & $t_1$ & $t_2-t_1$\\
				&&&\vdots\\
				$f_j$ & $f_k$ & $t_j$ & $t_k-t_j$\\
				&&&\vdots\\
				$f_m$ & $f_n$ & $t_m$ & $t_n-t_m$\\
			\end{tabular}
		\end{latin}
	\end{center}
	
	زمان شروع کلیپ در آهنگ مربوطه مشخص نیست. زمان‌های $t_j$ در جدول کلیپ وابسته به شروع کلیپ هستند. کلیپ با آفستی نامعلوم از ابتدای آهنگ شروع می‌شود. برای پیدا کردن مطابقت با کلیپ در پایگاه داده باید مطابقتی بین جدول (دیاگرام تصمیم گیری) کلیپ با جدول آهنگ‌های موجود در پایگاه داده پیدا کنیم. این کار را با لغزاندن جدول کلیپ روی جدول هر آهنگ برای پیدا کردن جایی که نقاط زیادی مطابق باشند، انجام می‌دهیم.
	
	استفاده از جفت پیک‌ها به عنوان ویژگی سه مقدار مستقل از آفست کلیپ به ما می‌دهد: $(f_1, f_2, t_2-t_1)$. آهنگی که بیشترین سه تایی $(f_1, f_2, t_2 - t_1)$ مشترک با کلیپ داشته باشد، به احتمال زیاد منبع کلیپ است.
\end{document}
